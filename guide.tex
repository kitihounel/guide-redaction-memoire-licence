\documentclass[12pt]{article}
\usepackage[frenchb]{babel}
\usepackage{a4wide}
\usepackage{caption}
\usepackage{fontspec}
\usepackage{newpxtext,newpxmath}
\usepackage[hidelinks]{hyperref}

% Title Page
\title{Conseils pour la rédaction d'un mémoire de licence en SIL}
\author{Lionel \textsc{Kitihoun}}
\date{Version d'avril 2020}

% To change default table caption.
\captionsetup[table]{name=Tableau}

\setlength{\parskip}{0.7em}
\renewcommand{\baselinestretch}{1.2}
\setlength{\intextsep}{1em}

\begin{document}
\maketitle

\section*{Introduction}
Ce document contient quelques conseils et indications pour aider les étudiants dans la rédaction de leur mémoire. Il ne s'agit aucunement de recommandations ou de directives à suivre à la lettre. L'étudiant devra faire des choix en fonction de sa situation et des directives de son superviseur.

\section{Généralités}

\subsection{Remarques préliminaires et erreurs à éviter}
\begin{itemize}
  \item Le mémoire se rédige au cours de  l'exécution du projet, et non une fois que l'application ou le travail est terminé.
  \item Il ne faut pas attendre de finir de rédiger le mémoire et aller voir le maître de mémoire pour qu'il le valide et signe. Il convient de faire régulièrement des rapports au superviseur pour l'informer de l'évolution du travail et obtenir de l'aide en cas de besoin.
  \item Vous devez avec l'aide de votre maître de stage évaluer le projet afin de planifier son exécution. Planifier votre travail augmente vos chances de réussite et vous permettra de régulièrement faire des bilans sur la bonne avancée de votre projet.
  \item Il ne faut pas croire que produire un gros document signifie que l'on a fait un bon travail et que cela impressionnera le jury. Vous êtes garant du contenu de votre mémoire et devez justifier chaque choix et affirmation qu'il contient. Cela veut dire qu'il ne faut pas copier le contenu d'autres documents sans comprendre ou vérifier ce qui y est dit.
  \item Il est souvent préférable de présenter les longs extraits de code en annexe.
\end{itemize}

\subsection{\`A propos du mémoire}
Le mémoire est un document important. Il est un héritage laissé par son auteur aux promotions suivantes. Il reflète le travail effectué et sert à se faire une opinion sur son auteur. Un mémoire bien rédigé avec un contenu de qualité donne une bonne impression et valorise son auteur.

Le mémoire présente le travail effectué lors d'un projet. Il détaille les objectifs à atteindre, les choix effectués, les résultats obtenus et présente les améliorations futures à apporter.

\subsection{Quelques conseils sur la présentation du document}
Pour les recommandations sur la manière de bien rédiger des documents en langue française, nous conseillons vivement la lecture du manuel \textit{Petites leçons de typographie} de M. Jacques \textsc{André}. Une copie du manuel accompagne normalement ce document. Il est disponible en ligne à l'adresse \url{http://jacques-andre.fr/ed/index.html\#lessons}.

Le mémoire doit être bien présenté et lisible. Faites usage des outils de correction orthographique intégré aux logiciels de traitement de texte pour éviter au maximum les fautes d'orthographe ou de grammaire. Faites aussi relire le document par vos proches et vos aînés pour avoir leurs avis et conseils. 

N'abusez pas des polices de caractères dans le document. Il est préférable de n'utiliser que deux polices pour tout le document : une police de caractères avec empattements (\textit{serif}) pour le corps et une police sans empattement (\textit{sans serif}) pour les titres. Si vous devez présenter des extraits de code, utilisez une police à chasse fixe (\textit{monospace}). Veuillez également à ce que la taille du texte soit d'au moins 12 points pour permettre une lecture aisée.

Les figures, tableaux et diagrammes présentés dans le document doivent être lisibles. Si une page contient une figure assez large, n'hésitez pas à mettre la page en paysage.

\section{Structuration du mémoire}

\subsection{Introduction}
L'introduction part d'un cadre général pour aboutir à la problématique abordée dans le mémoire. Il convient d'éviter les introductions commençant par \og L'informatique est la science du traitement rationnel et automatique de l'information \fg{} ou encore \og Il n'est plus à démontrer que toute entreprise a besoin de l'outil informatique pour un fonctionnement efficace \fg{}.

L'introduction présente le sujet abordé dans le mémoire. \`A la fin de sa lecture, on doit avoir une idée du travail effectué et avoir envie ou non de continuer la lecture du document.

\subsection{Première partie : présentation du maître d'\oe{}uvre et du projet}
Cette première partie contient généralement deux à trois chapitres. Un premier chapitre présente le maître d'\oe{}uvre, c-à-d la structure bénéficiaire du projet. Vous aurez à parler des activités de la structure, de son organisation et de ses ambitions.

Un deuxième chapitre présente dans une première partie les besoins qui ont donné naissance au projet sur lequel vous avez travaillé et ensuite décrit brièvement les étapes suivies lors de l'exécution du projet ainsi que les méthodes et outils que vous avez utilisés. Il s'agit entre du langage de modélisation\footnote{En licence, UML est généralement utilisé}, du SGBD assurant la persistance des données, du langage programmation et de l'EDI\footnote{Environnement de développement intégré} utilisés. Vous devez aussi parler de la méthodologie de développement logiciel choisie pour l'exécution du projet et justifier ce choix.

Un dernier chapitre facultatif est consacré à un état de l'art. L'état de l'art présente les travaux effectués par d'autres personnes et qui sont en rapport avec votre projet. Il est presque certain que d'autres avant vous ont eu à travailler sur des sujets similaires. Il existe aussi sur le marché des logiciels qui sont sans doute adaptés au besoin pour lequel vous développez une solution. Vous devez donc faire un bilan de ces travaux antérieurs et des informations que vous avez obtenues grâce à eux.

\subsection{Deuxième partie : modélisation UML}
Cette partie détaille le système que vous avez conçu. Les deux diagrammes centraux sont le diagramme des cas d'utilisation et le diagramme des classes. Les diagrammes de séquence, d'activité, de paquetage et de déploiement doivent aussi être présents. Un ou plusieurs diagrammes d'objets peuvent aussi accompagner le diagramme des classes pour en illustrer certains aspects.

Les diagrammes UML sont regroupés suivant trois axes de modélisation.

\begin{table}[!h]
  \begin{tabular}{|l|l|}
    \hline
    \textbf{Axe} & \textbf{Diagrammes} \\
    \hline
    Fonctionnel & Diagramme des cas d'utilisation \\
    \hline
    Statique & Diagramme des classes, diagramme de paquetage, diagramme d'objets,\\
             & diagramme de déploiement \\
    \hline
    Dyamique & Diagrammes de séquence, diagrammes d'activité \\
    \hline
  \end{tabular}
  \caption{Axes de modélisation UML}
\end{table}

Vous pouvez donc consacrer un chapitre à chacun des ces axes de modélisation.

\subsubsection{Modélisation fonctionnelle}
Il s'agira dans ce chapitre de présenter les différents cas d'utilisation identifiés et le diagramme des cas d'utilisation. Ensuite vous devez procéder à la description des cas d'utilisation les plus pertinents de votre système. Il en faut au moins quatre. La description d'un cas d'utilisation peut être accompagnée de l'IHM qui permet sa réalisation.

\subsubsection{Modélisation statique}
Ce chapitre est en grande partie consacré au diagramme des classes. Avant de présenter ce diagramme, il convient de décrire les différentes classes identifiées suite à votre analyse en expliquant leur utilité. Si vous sentez le besoin d'illustrer certains aspects de votre diagramme des classes pour en faciliter la compréhension, vous pouvez avoir recours au diagramme d'objets. Il faudra veiller à présenter également le diagramme de paquetage.

\subsubsection{Modélisation dynamique}
Vous présentez ici les diagrammes de séquence et d'activité. Il est préférable de présenter les diagrammes de séquence relatifs aux cas d'utilisation que vous avez eu à décrire dans le chapitre consacré à la modélisation fonctionnelle. Les diagrammes d'activité doivent également présenter des fonctionnalités pertinentes du système.

\subsection{Troisième partie : détails pratiques sur la réalisation de l'application}
Cette partie est consacrée à certains aspects de processus développement d'un logiciel qui ne sont pas abordés par UML. Il s'agit entre autres des mesures prises pour assurer la sécurité du système, les différents profils d'utilisateur \footnote{Aussi bien au niveau de l'application que de la base de données.}, les sauvegardes de la base de données, la création et la gestion des utilisateurs, les procédures de mise à jour de l'application, etc. 

Pour ce qui est du déploiement de l'application, vous pouvez ajouter un diagramme de déploiement pour présenter les différents composants du système.

\subsection{Bibliographie et annexes}
Vous devez présenter une bibliographie claire avec des références actualisées. Pour les pages web, il convient d'indiquer les dates de consultation. Pour la présentation des références bibliographiques, consulter encore une fois les \textit{Petites leçons de typographie} de M. Jacques \textsc{André}.

Les annexes sont destinées à tous les documents et informations complémentaires que vous jugez utiles d'ajouter au mémoire. Il peut s'agir d'extraits de code que vous jugez important de présenter ou des documents en rapport avec les activités du maître d'ouvre, comme des factures ou des textes de réglementation.

\subsection{Conclusion}
Vous faites un résumé du travail effectué, et parlez des améliorations à apporter à votre application. Vous pouvez également parler des acquis que vous a apportés la réalisation du projet présenté dans le mémoire.

\end{document}
